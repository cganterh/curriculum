\documentclass[11pt,letterpaper,sans]{moderncv}        % possible options include font size ('10pt', '11pt' and '12pt'), paper size ('a4paper', 'letterpaper', 'a5paper', 'legalpaper', 'executivepaper' and 'landscape') and font family ('sans' and 'roman')

% moderncv themes
\moderncvstyle{fancy}                             % style options are 'casual' (default), 'classic', 'banking', 'oldstyle' and 'fancy'
\moderncvcolor{purple}                               % color options 'black', 'blue' (default), 'burgundy', 'green', 'grey', 'orange', 'purple' and 'red'
\nopagenumbers{}                                  % uncomment to suppress automatic page numbering for CVs longer than one page

% character encoding
\usepackage[utf8]{inputenc}                       % if you are not using xelatex ou lualatex, replace by the encoding you are using

% adjust the page margins
\usepackage[scale=0.85]{geometry}
\setlength{\hintscolumnwidth}{3.75cm}                % if you want to change the width of the column with the dates
%\setlength{\makecvtitlenamewidth}{10cm}           % for the 'classic' style, if you want to force the width allocated to your name and avoid line breaks. be careful though, the length is normally calculated to avoid any overlap with your personal info; use this at your own typographical risks...

% personal data
\name{Cristóbal}{Ganter}
% \title{Curriculum Vitae}                               % optional, remove / comment the line if not wanted
\phone[mobile]{+56~9~7968~7919}                   % optional, remove / comment the line if not wanted; the optional "type" of the phone can be "mobile" (default), "fixed" or "fax"
\email{cganterh@gmail.com}                               % optional, remove / comment the line if not wanted
\homepage{www.cganterh.net}                         % optional, remove / comment the line if not wanted
\social[linkedin]{cganterh}                        % optional, remove / comment the line if not wanted
\social[github]{cganterh}                              % optional, remove / comment the line if not wanted


\begin{document}

\makecvtitle

\section{Languages}
    \cvitem{Spanish}{First language}
    \cvitem{English}{Advanced}
    \cvitemwithcomment{German}{Intermediate}{Sprachdiplom level 2 approved.}

\section{Education}
\cventry{2006--2016}{Electronic Engineer}{Universidad Técnica Federico Santa María}{Valparaíso}{}{Majors in Computer Systems and Informatics}  % arguments 3 to 6 can be left empty
\cventry{1994--2005}{Pre-University Schooling}{Colegio Alemán de Santiago}{Santiago}{}{}  % arguments 3 to 6 can be left empty

% \section{Master thesis}
% \cvitem{title}{\emph{Title}}
% \cvitem{supervisors}{Supervisors}
% \cvitem{description}{Short thesis abstract}

\section{Experience}
    \cventry{2015}{Freelancer}{Reiser Technologies}{Valparaíso}{}{}

    \cventry{2014}{MongoDB Consultant}{Xompass}{Valparaíso}{}{
        Advice was given to Xompass to optimize their
        MongoDB data base.
    }

    \cventry{2013}{Especialista en Computación}{Proyecto FIC SURHGE}{Valparaíso}{}{
        Activities:
        \begin{itemize}
            \item Linux system administration.
            \item Data analysis.
            \item Web application programming.
            \item Mathematical models programming.
        \end{itemize}
    }

    \cventry{2011--2012}{Ayudante de Cátedra}{UTFSM}{Valparaíso}{}{
        Se trabaja como ayudante del ramo
        \emph{Estructura de Datos y Algoritmos}.\newline{}
        Actividades realizadas:
        \begin{itemize}
            \item Ayudantías presenciales.
            \item Corrección de pruebas y tareas.
            \item
                Elaboración de material para taller de
                programación en C.
            \item Elaboración de preguntas para pruebas.
        \end{itemize}
    }

    \cventry{2010--2011}{Ayudante de Desarrollo de Software}{UTFSM}{Valparaíso}{}{
        Actividades realizadas:
        \begin{itemize}
            \item Administración de sistemas Linux.
            \item
                Desarrollo de software relacionado con
                probabilidades en Delphi.
            \item
                Puesta en marcha y mantenimiento de un
                sistema de control de versiones.
        \end{itemize}
    }

    \cventry{2009}{Practicante}{Wetland}{Santiago}{}{
        Se programan PLCs para su uso en plantas de
        tratamiento de aguas.
    }

    \cventry{2005}{Empleado}{CONAC}{Talca}{}{
        Se diseña un controlador electrónico de nivel de
        estanque.
    }

    \cventry{2002}{Encargado de Aseo}{Camping Radal}{Reserva Nacional Radal Siete Tazas}{}{}

\section{Conocimientos de Computación}
\subsection{Sistemas Operativos}
    \begin{cvcolumns}
        \cvcolumn{Advanced}{Linux}
        \cvcolumn{Intermediate}{Windows}
        \cvcolumn{Beginner}{BSD}
    \end{cvcolumns}

\subsection{Lenguajes de Programación}
    \begin{cvcolumns}
        \cvcolumn{Advanced}{
            C\newline{}
            CoffeeScript\newline{}
            Python
        }
        \cvcolumn{Intermediate}{
            AWK\newline{}
            Delphi\newline{}
            Java\newline{}
            JavaScript\newline{}
            Pascal
        }
        \cvcolumn{Beginner}{
            Bash\newline{}
            C++\newline{}
            Ladder\newline{}
            MATLAB\newline{}
            MIPS\newline{}
            PHP\newline{}
            Visual Basic
        }
    \end{cvcolumns}

\subsection{Otros Lenguajes}
    \begin{cvcolumns}
        \cvcolumn{Advanced}{
            CSS\newline{}
            HTML5\newline{}
            JSON\newline{}
            make\newline{}
            Markdown\newline{}
            SCSS
        }
        \cvcolumn{Intermediate}{
            SVG\newline{}
            Verilog\newline{}
            XML\newline{}
            YAML\newline{}
        }
        \cvcolumn{Beginner}{
            SQL
        }
    \end{cvcolumns}

\subsection{Frameworks}
    \begin{cvcolumns}
        \cvcolumn{Advanced}{
            Tornado Web
        }
        \cvcolumn{Beginner}{
            Django\newline{}
            Flask\newline{}
            Ruby on Rails
        }
    \end{cvcolumns}

\subsection{Herramientas}
    \begin{cvcolumns}
        \cvcolumn{Advanced}{
            GIT\newline{}
            Inkscape\newline{}
            LibreOffice\newline{}
            Microsoft Office\newline{}
            GNUCash
        }
        \cvcolumn{Intermediate}{
            GIMP
        }
    \end{cvcolumns}

\section{Intereses}
    \cvlistitem{Astronomía}
    \cvlistitem{Ciencia}
    \cvlistitem{Cocina}
    \cvlistitem{Programación}

\end{document}
